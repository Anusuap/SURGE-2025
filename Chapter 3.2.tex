\documentclass{article}
\usepackage{graphicx}    % Required for inserting images
\usepackage{amsmath}     % For better math formatting
\usepackage{amssymb}     % For additional math symbols
\usepackage{mathrsfs}    % For script letters like \mathscr
\usepackage{amsfonts} % For symbols like \mathbb
\usepackage{amsthm}

\title{Chapter 3.2}
\author{Anusua Paul}
\date{June, 2025}

\numberwithin{equation}{section}


\newtheorem{theorem}{Theorem}[section]
\newtheorem{corollary}[theorem]{Corollary}
\newtheorem{lemma}[theorem]{Lemma}
\newtheorem{proposition}[theorem]{Proposition}
\newtheorem{remark}[theorem]{Remark}
\newtheorem{note}[theorem]{Note}
\newtheorem{definition}[theorem]{Definition}
\newtheorem{notation}[theorem]{Notation}
\newtheorem{example}[theorem]{Example}


\newcommand{\F}{\mathscr{F}}
\newcommand{\pp}{\mathbb{P}}

\begin{document}

\maketitle

\section{Recurrence and Transience of Brownian Motion}

Suppose 
\[
B_t(\omega) : (\Omega, \F, \pp) \times [0, \infty) \to (\mathbb{R}^d, \mathscr{B}(\mathbb{R}^d))
\]
is a Brownian motion in dimension \( d \). It is called \textbf{transient} if
\[
\lim_{t \to \infty} |B_t(\omega)| = \infty \quad \text{almost surely},
\]
that is,
\[
\pp\left( \left\{ \omega \in \Omega : \lim_{t \to \infty} |B_t(\omega)| < \infty \right\} \right) = 0.
\]
In particular, for fixed \( \omega \in \Omega \), i.e., for a fixed path, we often write \( B_t(\omega) = B(t) \).
Note that  \(
\left\{ \omega :\lim_{t \to \infty} |B_t(\omega)| = \infty \right\}
\) is a \textit{tail event} (as Brownian motion has independence increament property so this event belongs to tail sigma algebra), and hence by the zero-one law for tail events, it happens either with probability 0 or with probability 1. Thus, for fixed \( d \), the Brownian motion is either transient or recurrent.
\begin{definition}
Define \( \mathscr{G}(t) = \sigma(B(s) : s \geq t) \). The \textbf{tail \(\sigma\)-algebra} of Brownian motion  is defined as
\[
\mathscr{T} = \bigcap_{t \geq 0} \mathscr{G}(t),
\]
and consists of all tail events.
\end{definition}
\begin{theorem}[{\cite[Theorem 2.9]{PeresMortersBook}}]
\textbf{Zero-One Law for Tail Events : } Let \( x \in \mathbb{R}^d \) and let \( A \in \mathscr{T} \) be a tail event. Then,
\[
\mathbb{P}_x(A) \in \{0, 1\}.
\]
\end{theorem}
\noindent In this section, we are interested in determining for which dimensions \( d \) Brownian motion is transient, and for which dimensions it is recurrent. This question is closely related to the exit probabilities of a Brownian motion from an annulus.
Consider the annulus
\[
A = \left\{ y \in \mathbb{R}^d : r < |y| < R \right\}, \quad \text{for } 0 < r < R < \infty.
\]
Suppose a Brownian motion starts at a point \( x \in A \). We are interested in the probability that the motion hits \( \partial \mathbb{B}(0, r) \) before it hits \( \partial \mathbb{B}(0, R) \). 
Now, as \( R \to \infty \) and \( t \to \infty \), this event becomes equivalent to the event 
\[
\left\{\omega : \lim_{t \to \infty} |B_t(\omega)| = \infty \right\},
\]
which is precisely the event of transience.
The answer to this exit probability is given in terms of a harmonic function defined on the annulus, and is therefore closely related to the \textbf{Dirichlet Problem}. \\
To find an explicit solution \( u : \overline{A} \to \mathbb{R} \) of the Dirichlet problem on an annulus, we first assume that \( u \) is \textbf{spherically symmetric}, i.e., the value of the function depends only on the norm distance \( |x| \) from the origin and not on the angular position. Since the annulus has a symmetric structure, this is a reasonable assumption.
If \( u \) is spherically symmetric, then there exists a function \( \psi : [r, R] \to \mathbb{R} \) such that \(u(x) = \psi(|x|^2).\) \\
We now compute the derivatives of \( u \) in terms of \( \psi \). Letting \( y = |x|^2 \geq 0 \) , we have:
\[
\partial_{i} u(x) = \frac{d}{dx_i} \psi(|x|^2) = \psi'(|x|^2) \cdot 2x_i,
\]
and
\[
\partial_{ii}^2 u(x) = \frac{d^2}{dx_i^2} \psi(|x|^2) = \psi''(|x|^2) \cdot 4x_i^2 + 2\psi'(|x|^2).
\]
Therefore, the Laplacian of \( u \) is
\[
\Delta u(x) = \sum_{i=1}^d \left( 4x_i^2 \psi''(y) + 2\psi'(y) \right).
\]
Now, using the identity \( \sum_{i=1}^d x_i^2 = |x|^2 = y \), we get
\[
\Delta u(x) = 4y \psi''(y) + 2d \psi'(y).
\]
Setting \( \Delta u = 0 \) for harmonicity, we obtain:
\[
4y \psi''(y) + 2d \psi'(y) = 0.
\]
Dividing through by 2, we get:
\begin{align*}
    2y \, \psi''(y) + d \, \psi'(y) &= 0 \\
    \text{and hence , } \psi''(y) + \frac{d}{2y} \, \psi'(y) &= 0
\end{align*}

Here the integrating factor is given by, 

\[
\text{IF} = \exp\left( \int \frac{d}{2y} \, dy \right) = y^{d/2}
\]

Now multiplying with IF we get:

\[
\psi''(y) \, y^{d/2} + \frac{d}{2} \, y^{\frac{d}{2}-1} \, \psi'(y) = 0
\]

\[
\text{and hence , }\frac{d}{dy} \left( \psi'(y) \, y^{d/2} \right) = 0
\]

\[
\text{so} \quad \psi'(y) \, y^{d/2} = \text{const}
\]

\[
\text{and hence , } \psi'(y) = \text{const} \cdot y^{-d/2}
\]
\noindent Thus, \( \Delta u = 0 \) holds for \( |x| > 0 \) as well.
Hence, the general solution \( u(x) \) is given by :
\begin{equation}\label{all-harmonic}
u(x) = 
\begin{cases}
k_1|x| & \text{if } d = 1, \\
k_2\log |x| & \text{if } d = 2, \\
k_d|x|^{2 - d} & \text{if } d \geq 3.
\end{cases}
\end{equation}
where \(k_d \) are constants. Note that for \(d \leq 2\) u(x) is increasing in x and for \(d \geq 3\) onwards it is decreasing in x.\\
Write \( u(r) \) for the value of \( u(x) \) for all \( x \in \partial \mathbb{B}(0, r) \). As \( u(\cdot) \) is spherically symmetric and only depends on the norm distance of \( x \), this is a reasonable notation. Define the stopping times
\[
T_r = \tau(\partial \mathbb{B}(0, r)) = \inf \{ t > 0 : |B(t)| = r \} \quad \text{for } r > 0,
\]
and \( T = T_r \wedge T_R \) denotes  the first exit time from the annulus \( A \). Here, assume that \( T \) is finite almost surely, that is,
\[
\mathbb{P}\left( \left\{ \omega \in \Omega : T(\omega) = \infty \right\} \right) = 0.
\]
From equation \eqref{all-harmonic}, \( u(x) \) is continuous on the annulus, and since every boundary point satisfies the Poincaré cone condition, by Dirichlet Problem , we have
\[
u(x) = \mathbb{E}_x \left[ u(B(T)) \right] = u(r) \, \mathbb{P}_x(T_r < T_R) + u(R)\left(1 - \mathbb{P}_x(T_r < T_R)\right).
\]
From this we can solve for the exit probability
\[
\mathbb{P}_x(T_r < T_R) = \frac{u(R) - u(x)}{u(R) - u(r)},
\]
and this gives the explicit solution to the exit problem.
\begin{theorem}\label{exit-annulus}
[{\cite[Theorem 3.18]{PeresMortersBook}}]
Let \( \{B(t) : t \geq 0\} \) is a Brownian motion in dimension \( d \geq 1 \) started at \( x \in A := \{y \in \mathbb{R}^d : r \leq |y| \leq R \} \), inside an annulus \( A \) with radii \( 0 < r < R < \infty \). Then,
\begin{equation}
\mathbb{P}_x(T_r < T_R) =
\begin{cases}
\displaystyle \frac{R - |x|}{R - r} & \text{if } d = 1, \\[10pt]
\displaystyle \frac{\log R - \log |x|}{\log R - \log r} & \text{if } d = 2, \\[10pt]
\displaystyle \frac{R^{2 - d} - |x|^{2 - d}}{R^{2 - d} - r^{2 - d}} & \text{if } d \geq 3.
\end{cases}
\end{equation}

\end{theorem}


\begin{proof}
The proof directly follows from equation (\ref{all-harmonic}).
\end{proof}
\noindent Letting \( R \uparrow \infty \) in Theorem \ref{exit-annulus} leads to the following corollary.

\begin{corollary}[{\cite[Corollary 3.19]{PeresMortersBook}}]
For any \( x \notin \mathbb{B}(0, r) \), we have
\[
\mathbb{P}_x(T_r < \infty) =
\begin{cases}
1 & \text{if } d \leq 2, \\[10pt]
\displaystyle \left( \frac{r}{|x|} \right)^{d - 2} & \text{if } d \geq 3.
\end{cases}
\]
\end{corollary}
\begin{definition}
Suppose 
\[
X_t(\omega) : (\Omega, \mathscr{F}, \mathbb{\mu}) \times [0, \infty) \to (\mathbb{R}^d, \mathscr{B}(\mathbb{R}^d))
\] is a Markov process . For fixed \(\omega \in \Omega\) write \(X_t(\omega) = X(t)\) .
\begin{itemize}
    \item \textbf{Point recurrent} : if for every \( x \in \mathbb{R}^d \), almost surely, there exists a (random) sequence \( t_n \uparrow \infty \) such that \( X(t_n) = x \) for all \( n \in \mathbb{N} \).
    
    \item \textbf{Neighbourhood recurrent} : if for every \( x \in \mathbb{R}^d \) and \( \varepsilon > 0 \), almost surely, there exists a (random) sequence \( t_n \uparrow \infty \) such that \( X(t_n) \in B(x, \varepsilon) \) for all \( n \in \mathbb{N} \).
    
    \item \textbf{Transient} : if \( |X(t)| \to \infty \) almost surely as \( t \to \infty \).
\end{itemize}
\end{definition}
\begin{theorem}[{\cite[Theorem 3.20]{PeresMortersBook}}]
Brownian motion is
\begin{itemize}
    \item \textbf{point recurrent} in dimension \( d = 1 \),
    \item \textbf{neighbourhood recurrent} but not point recurrent, in dimension \( d = 2 \),
    \item \textbf{transient} in dimension \( d \geq 3 \).
\end{itemize}
\end{theorem}
\begin{remark}[{\cite[Remark 3.21]{PeresMortersBook}}]
\textit{Neighbourhood recurrence}, in particular, implies that the path of a planar Brownian motion (running for an infinite amount of time) is dense in the plane. It means that for any \(x \in R^2\) and for any \( \epsilon > 0\) the Brownian motion will enter in \(B(x,\epsilon)\) infinitely often almost surely.
\end{remark}
\begin{lemma}
\textbf{Paley-Zygmund Inequality : } [{\cite[Lemma 3.23]{PeresMortersBook}}]
For any nonnegative random variable \( X \) with \( \mathbb{E}[X^2] < \infty \),
\[
\mathbb{P}(X > 0) \geq \frac{(\mathbb{E}[X])^2}{\mathbb{E}[X^2]}.
\]
\end{lemma}
\begin{proof}
The Cauchy–Schwarz inequality gives
\[
\mathbb{E}[X] = \mathbb{E}[X \cdot \mathbf{1}_{\{X > 0\}}] \leq \left(\mathbb{E}[X^2]\right)^{1/2} \left(\mathbb{P}(X > 0)\right)^{1/2},
\]
and the required inequality follows immediately by squaring both sides:
\[
\mathbb{P}(X > 0) \geq \frac{(\mathbb{E}[X])^2}{\mathbb{E}[X^2]}.
\]
\end{proof}
\begin{lemma}[{\cite[Lemma 3.24]{PeresMortersBook}}]
Suppose \( E_1, E_2, \ldots \) are events such that
\[
\sum_{n=1}^\infty \mathbb{P}(E_n) = \infty
\quad \text{and} \quad
\liminf_{k \to \infty} \frac{\sum_{m=1}^k \sum_{n=1}^k \mathbb{P}(E_n \cap E_m)}{\left( \sum_{n=1}^k \mathbb{P}(E_n) \right)^2} < \infty.
\]
Then, with positive probability, infinitely many of the events take place, that is,
\[
\mathbb{P} \left( \limsup_{n \to \infty} E_n \right) > 0.
\]
\end{lemma}



\bibliographystyle{plain}
\bibliography{references}



\end{document}
